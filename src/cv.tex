%%%%%%%%%%%%%%%%%%%%%%%%%%%%%%%%%%%%%%%%%
% XeLaTeX Template:
%   https://www.latextemplates.com/template/freeman-cv
%
% Original authors:
%   Vel (vel@LaTeXTemplates.com)
%   Alessandro Plasmati

% Modified by:
%   Matteo Giorgi
%
% License:
%   CC BY-NC-SA 3.0 (http://creativecommons.org/licenses/by-nc-sa/3.0/)
%%%%%%%%%%%%%%%%%%%%%%%%%%%%%%%%%%%%%%%%%

%----------------------------------------------------------------------------------------
%	PACKAGES AND OTHER DOCUMENT CONFIGURATIONS
%----------------------------------------------------------------------------------------

\documentclass[10pt]{article} % Font size, can be: 10pt, 11pt or 12pt
\usepackage{paralist}
\usepackage{hyperref}

\input{structure.tex} % Include the file that specifies the document structure

% Headers and footers can be added with the \lhead{} \rhead{} \lfoot{} \rfoot{} commands
% Example right footer:
%\rfoot{\color{headings}{\sffamily Last update: \today. Typeset with Xe\LaTeX}}

%----------------------------------------------------------------------------------------

\begin{document}

\vspace*{0.6\baselineskip}

\begin{paracol}{2} % Begin the multi-column environment

%----------------------------------------------------------------------------------------
%	NAME AND CURRICULUM VITAE TEXT
%----------------------------------------------------------------------------------------

\parbox[top][0.12\textheight][c]{\linewidth}{ % Parbox to hold the author name and CV text; fixed height to match the coloured box to the right, centred vertically and full line width
	\vspace{-0.04\textheight} % Reduce whitespace above the parbox to separate it from the main content
	\centering % Centre text
%	{\sffamily\Huge Matteo Giorgi}\\\medskip % Your name
%	{\Huge\color{headings}\cvtextfont Curriculum Vitae}
    {\sffamily\Huge Matteo Giorgi}\\
    {\Large\color{headings} Curriculum Vit\ae}
}

%----------------------------------------------------------------------------------------
%	WORK EXPERIENCE
%----------------------------------------------------------------------------------------

%\section{Formazione}
%
%% Blank \workposition command to add another job:
%
%%\workposition{} % Duration
%%{} % FT/PT (full time or part time)
%%{} % Employer
%%{} % Job title
%%{} % Description
%
%% All 5 parameters must be supplied but any can be empty if you don't need them
%
%%------------------------------------------------
%
%\workposition{Current, from Jan 1995} % Duration
%{} % FT/PT (full time or part time)
%{} % Employer
%{} % Job title
%{As part of this promotion, I began conducting nuclear and subatomic research in the Anomalous Materials department. My team and I are particularly interested in dimensionality and its interaction with spacetime. The focus is on practical outcomes and applications in teleportation and communication with distal locations.} % Description
%
%%------------------------------------------------
%
%\workposition{Feb 1991 -- Jan 1995} % Duration
%{} % FT/PT (full time or part time)
%{} % Employer
%{} % Job title
%{This position involved transitioning from purely theoretical work to experimental applications utilising the immense resources of Black Mesa. The transition required an initial learning curve in hazard containment, health and safety procedures and operating experimental infrastructure. Manipulating valves, carts, buttons, levers, etc considerably increased my physical fitness.}  % Description

\section{Bio \& Education}

\begin{supertabular}{rl} % Start a table with two columns, the table will ensure everything is aligned
	
	%------------------------------------------------
	
	\tableentry{Birth}{Born in Pisa (Italy), 18 June 1987}{spaceafter}

    \tableentry{High school}{Scientific high school diploma,}{}
    \tableentry{}{\textit{Liceo Scientifico F. Enriques} (Livorno - Italy)}{spaceafter}

    \tableentry{University}{Studies in Mechanical Eng. at the Department}{}
    \tableentry{}{of Mechanical Engineering, \textit{University of Pisa}}{spaceafter}
    \tableentry{}{Currently finishing B.Sc. in Computer Science}{}
    \tableentry{}{at the \textit{University of Pisa}}{}
	
	%------------------------------------------------
	
\end{supertabular}

%\bigskip % Extra whitespace before the next section
\vspace{15pt}

%----------------------------------------------------------------------------------------
%	SOFTWARE
%----------------------------------------------------------------------------------------

\section{Areas of Interest}

% Example \longformdescription{} command to add another section:

%\longformdescription{Heading}{Description}

%------------------------------------------------

%\longformdescription{Passionate}{I have been interested in theoretical physics such as quantum mechanics and relativity from an early age. My education and research have cemented this interest into a passion. I greatly enjoy carrying out fundamental physics research with potential practical applications.}

I am very keen on programming languages and I often find myself playing with a new language trying to improve my skills; this has made me focused on \textit{Functional Programming}, experimenting with languages such as \textit{Ocaml}, \textit{Haskell} and \textit{Scala}.

\medskip
I am interested in \textit{Cryptanalysis} too, spending some time in the field of \textit{Lattice Theory}, because of its implications regarding elliptic-curve and post-quantum cryptography.

% \medskip % Extra whitespace before the next section

%----------------------------------------------------------------------------------------

% \vspace{8pt}

%----------------------------------------------------------------------------------------

\switchcolumn % Switch to the next paracol column

%----------------------------------------------------------------------------------------
%	COLOURED CONTACT DETAILS BOX
%----------------------------------------------------------------------------------------

\parbox[top][0.12\textheight][c]{\linewidth}{ % Parbox to hold the colour box; fixed height to match the name/CV text to the left, centred vertically and full line width
	\vspace{-0.04\textheight} % Reduce whitespace above the parbox to separate it from the main content
	\colorbox{shade}{ % Create the coloured box
		\begin{supertabular}{p{0.05\linewidth}|p{0.775\linewidth}} % Start a table with two columns, the table will ensure everything is aligned
			\raisebox{-1pt}{\faGlobe} & \href{https://www.geoteo.net}{www.geoteo.net} \\ % Website
            \raisebox{-1pt}{\faGithub} & \href{https://github.com/matteogiorgi}{github.com/matteogiorgi} \\ % GitHub
			\raisebox{-1pt}{\faEnvelopeSquare} & \href{mailto:matteo.giorgi@protonmail.com}{matteo.giorgi@protonmail.com} \\ % Email address
			\raisebox{-1pt}{\faPhoneSquare} & (+39) 3385311520 \\ % Phone number
		\end{supertabular}
	}
}

%----------------------------------------------------------------------------------------
%	EDUCATION
%----------------------------------------------------------------------------------------

%\section{???????} 
%
%% Blank \educationentry{} command to add another degree:
%
%%\educationentry{} % Duration
%%{} % Degree
%%{} % Honours, achievements or distinctions (e.g. first class honours)
%%{} % Department
%%{} % Institution
%
%% All 5 parameters must be supplied but any can be empty if you don't need them
%
%%------------------------------------------------
%
%\begin{supertabular}{rl} % Start a table with two columns, the table will ensure everything is aligned
%
%	%------------------------------------------------
%	
%	\educationentry{1986 -- 1990} % Duration
%	{Doctor of Philosophy} % Degree
%	{} % Honours, achievements or distinctions (e.g. first class honours)
%	{Theoretical Physics} % Department
%	{Massachusetts Institute of Technology} % Institution
%	
%	%------------------------------------------------
%	
%	\educationentry{1985} % Duration
%	{Master of Science} % Degree
%	{First Class Honours} % Honours, achievements or distinctions (e.g. first class honours)
%	{Theoretical Physics} % Department
%	{Massachusetts Institute of Technology} % Institution
%	
%	%------------------------------------------------
%	
%	\educationentry{1982 -- 1984} % Duration
%	{Bachelor of Physics} % Degree
%	{} % Honours, achievements or distinctions (e.g. first class honours)
%	{Department of Physics} % Department
%	{The University of Washington} % Institution
%	
%	%------------------------------------------------
%
%\end{supertabular}

%----------------------------------------------------------------------------------------
%	COMPUTER SKILLS
%----------------------------------------------------------------------------------------

\section{Programming \& Software}

% Example \tableentry{} command to add another line:

%\tableentry{Heading}{Content}{spaceafter}

% All 3 parameters must be supplied but any can be empty if you don't need them
% A "spaceafter" value in the third parameter will add some vertical space -- this is to be used between headings

%------------------------------------------------

\begin{supertabular}{rl} % Start a table with two columns, the table will ensure everything is aligned
	
	%------------------------------------------------
	
	\tableentry{Languages}{\textit{C/C++}, \textit{Java}, \textit{Python}, \textit{Javascript}, \textit{Go},}{}
    \tableentry{}{\textit{Ocaml}, \textit{Common Lisp}, \textit{Bash}}{spaceafter}

	\tableentry{UNIX}{\textit{GNU/Linux} and \textit{FreeBSD} system}{}
    \tableentry{}{management, \textit{C} system programming,}{}
    \tableentry{}{good knowledge of \textit{POSIX} standard}{spaceafter}

	\tableentry{Tools}{Experience with various software}{}
	\tableentry{}{such as \textit{Vim}, \textit{Emacs}, \textit{Git}, \textit{\LaTeX}}{}
	\tableentry{}{\textit{Matlab} and \textit{Wolfram Mathematica}}{spaceafter}
	
	\tableentry{Extra}{Knowledge of \textit{Sql}, \textit{HTML} and \textit{CSS}}{}

	%------------------------------------------------
	
\end{supertabular}

\vspace{15pt}

%----------------------------------------------------------------------------------------
%	COMMUNICATION SKILLS
%----------------------------------------------------------------------------------------

\section{Languages}

% Example \tableentry{} command to add another line:

%\tableentry{Heading}{Content}{spaceafter}

% All 3 parameters must be supplied but any can be empty if you don't need them
% A "spaceafter" value in the third parameter will add some vertical space -- this is to be used between headings

%------------------------------------------------

\begin{supertabular}{rl} % Start a table with two columns, the table will ensure everything is aligned
	
	%------------------------------------------------
	
	\tableentry{Italian}{Mother tongue}{spaceafter}

	\tableentry{English}{Advanced use in speaking and writing,}{}
	\tableentry{}{\textit{Cambridge C1 Advanced (IELTS 7.0)}}{spaceafter}
	
	%------------------------------------------------
    
    \tableentry{French}{Basic understanding in speaking and}{}
    \tableentry{}{writing, \textit{DELF A2} certification}{}
    
    %------------------------------------------------
	
\end{supertabular}
\end{paracol}
%----------------------------------------------------------------------------------------
%	SKILLS DESCRIPTION
%----------------------------------------------------------------------------------------

\vspace{15pt}
\section{Experiences \& projects}

% Example \longformdescription{} command to add another section:

%\longformdescription{Heading}{Description}

%------------------------------------------------

%\longformdescription{Goal Oriented}{I believe in action over long-winded discussions. I listen to everyone's viewpoints and use my judgement to immediately act based on consensus to achieve goals quickly and efficiently.}
%
%\longformdescription{Physical Dexterity}{Manual manipulation of experimental equipment and training within Black Mesa (e.g. the Hazard Course) have contributed to an enjoyment of working with my hands.}

%\begin{itemize}[\leftmargin=0cm,\labelsep=0cm]
%    \item[$\circ$] Supporto alla didattica come assistente di laboratorio \textit{C} dell'esame di Informatica tenuto dalla prof.ssa Pelagatti per il \textit{CdL} in \textit{Fisica} \textit{UniPi} (secondo semestre Anno accademico 2016-2017).
%    
%    \item[$\circ$] Esperienza da programmatore \textit{Python} alla costruzione plastico per non vedenti, realizzato al \textit{CNR} di Pisa sotto la supervisione del dott. Furfari e prof.ssa Pelagatti nel secondo trimestre 2017. Esecuzione in coppia con un collega laureando al \textit{CdL} in \textit{Informatica} \textit{UniPi}.
%\end{itemize}

\begin{supertabular}{rl} % Start a table with two columns, the table will ensure everything is aligned

	%------------------------------------------------
	
    \tableentry{CNR project}{Experience, together with a colleague of mine, as a \textit{Python} programmer inside the \href{http://www.area.pi.cnr.it}{\textit{CNR} offices} in Pisa, helping}{}
    \tableentry{}{in the realisation of a scale model for blind people. The project was under the supervision of Doctor Furfari from}{}
    \tableentry{}{\textit{CNR} and Professor Pelagatti from \textit{UniPi}.}{spaceafter}
	
	%------------------------------------------------
    
	\tableentry{Lab assistant}{Worked, together with Professor Pelagatti as lab assistant for the \href{http://didawiki.di.unipi.it/doku.php/fisica/informatica/201617/start}{\textit{C} laboratory exam} at \textit{B.Sc. in Physics}.}{spaceafter}
	
	%------------------------------------------------
    
	\tableentry{RSA report}{\href{https://github.com/matteogiorgi/wiener}{Independent studies}, under the supervision of Professor Romani from \textit{UniPi}, regarding the attacks that}{}
	\tableentry{}{exploit the \textit{RSA} cryptosystem vulnerabilities with a specific focus on the \textit{Wiener Attack} and its use of continued}{}
	\tableentry{}{fractions for the factorization of the \textit{RSA} module.}{spaceafter}
	
	%------------------------------------------------
    
	\tableentry{.udot}{\href{https://github.com/matteogiorgi/.udot}{Public repository} of all my \textit{GNU/Linux} scripts and config. files, written over time. The main purpose of the}{}
	\tableentry{}{project is to build a stable personal environment complete with all the tools needed to work efficiently.}{spaceafter}
	
	%------------------------------------------------
    
    \tableentry{Hackathon}{First place in the 2021 edition of \href{http://contaminationlab.unipi.it/conthackt-foodmobilitydigital}{\textit{ContHackt}}, organized by the \textit{University of Pisa} and \textit{Contamination Lab Pisa}. As}{}
    \tableentry{}{winner, our team had access to the 2021/2022 \href{https://eit-hei.eu/app/uploads/2021/10/EIT-Project-Fact-Sheet-EUAcceL.pdf}{\textit{EUAcceL}} project organized by the \textit{European Institute of Innovation}}{}
    \tableentry{}{\textit{and Technology} and won the final stage with a blockchain prototype for the tracking of food products.}{spaceafter}
	
	%------------------------------------------------
	
\end{supertabular}
%\medskip % Extra whitespace before the next section

%----------------------------------------------------------------------------------------

\vspace{15pt}
\section{Competitive sports}

More than ten years of \emph{sailing} activity at both national and international level in Optimist, 420 and 470 class, most of which among the ranks of the Junior National Team. Competed in several prestigious regattas, above all \href{https://www.kieler-woche.de}{\textit{Kiel Week}} and \href{https://www.barcolana.it}{\textit{Barcolana}} as the two largest sailing events in the world. Other major achievements:\\
% Winner of 1 national ranking list and qualified among the first ten italian juniores other four times.
% Competed, as a member of the italian national team, in 1 word championship (China), 3 european championships
% (Portugal, Ireland, Italy) and classified vice-champion in 1 european team championship (Germany).

% Other noteworthy achievements: winner of \textit{Young Barcolana}, \textit{Mediterranean Cup} and \textit{Trofeo Accademia Navale Jr}.\\
% Other prestigious competitions: \href{https://www.kieler-woche.de}{\textit{Kiel Week}} and \href{https://www.barcolana.it}{\textit{Barcolana}} as the two largest sailing event in the world.

\begin{supertabular}{rl} % Start a table with two columns, the table will ensure everything is aligned
	
	%------------------------------------------------
	
	\tableentry{National}{Ranked first once and qualified among the first ten Italian Juniors other four times. Among others, winner}{}
	\tableentry{}{of a \textit{Young Barcolana}, a \textit{Mediterranean Cup} and one \textit{Trofeo Accademia Navale Juniores}.}{}
    \tableentry{International}{Competed, as a member of the Italian National Team, in one World Championship (Qingdao - China),}{}
    \tableentry{}{three European Championships (Tavira - Portugal, Dublin - Ireland, Riva dal Garda - Italy) and classified}{}
    \tableentry{}{vice-champion in one European Team Championship (Berlin - Germany).}{}
	% \tableentry{}{Other noteworthy achievements: winner of \textit{Young Barcolana}, \textit{Mediterranean Cup} and \textit{Trofeo Accademia Navale Jr}.}{}
	
	%------------------------------------------------
    
    % \tableentry{Running}{I started as a middle-distance runner them moved to long distance, specialising in 10km and half-marathon.}{}
    % \tableentry{}{Sometime I took the time to do few ultra-marathon competitions, one over all the \textit{Etna Marathon}: an over 42km long}{}
    % \tableentry{}{race at 2000m msl on the volcano Etna.}{}
    
    %------------------------------------------------
	
\end{supertabular}

\end{document}
